\documentclass[onecolumn]{ctexart}
\usepackage[utf8]{inputenc}
\usepackage{amsmath}
\usepackage{amssymb}
\usepackage{amsthm}
\usepackage{mathtools}
\usepackage{geometry}
\usepackage{graphicx}
\usepackage{float}
\usepackage{xcolor}
\usepackage{listings}
\usepackage{indentfirst}
\usepackage{bm}
\usepackage{tikz}
\usetikzlibrary{shapes,arrows}
\geometry{a4paper,scale=0.8}

\newtheorem{definition}{Definition}
\newtheorem{theorem}{Theorem}
\newtheorem{proposition}{Proposition}
\newtheorem{lemma}{Lemma}
\newtheorem{corollary}{Corollary}
\newtheorem{remark}{Remark}
\newtheorem{example}{Example}

\title{Notes of "Ruled Quadric Surfaces"}
\author{Jinxin Wang}
\date{}

\begin{document}

\maketitle

\section{Overview}
Geometric interpretation of ruled surface: 由一簇直线构成的曲面叫做直纹面。

\section{Quadric Cylindrical Surface二次柱面}

Based on the geometric interpretation of cylindrical surfaces, they are a kind 
of ruled surface.

\section{Quadric Conical Surface二次锥面}

Based on the geometric interpretation of conical surfaces, they are a kind of 
ruled surface.

\section{Hypobolic Paraboloid双曲抛物面}

\subsection{Equation of Generatrix in Hypobolic Paraboloid}

Given the standard equation of a hypobolic paraboloid:
\begin{equation}
  \frac{x^2}{a^2} - \frac{y^2}{b^2} = 2z
\end{equation}
\[
  (\frac{x}{a} + \frac{y}{b})(\frac{x}{a} - \frac{y}{b}) = 2z
\]
Let $c = \frac{x}{a} + \frac{y}{b}$, then the line with the equation
\[
  l_c = 
  \begin{cases}
    \frac{x}{a} + \frac{y}{b} = c \\
    c(\frac{x}{a} - \frac{y}{b}) = 2z
  \end{cases}
\]
belongs to the hypobolic paraboloid.

Let $c = \frac{x}{a} - \frac{y}{b}$, then the line with the equation
\[
  l_c' = 
  \begin{cases}
    \frac{x}{a} - \frac{y}{b} = c \\
    c(\frac{x}{a} + \frac{y}{b}) = 2z
  \end{cases}
\]
belongs to the hypobolic paraboloid.

Therefore, we find two collection of straight generatrix in the hypoboloic 
paraboloid:
\begin{equation}
  \begin{split}
    I = \lbrace l_c | c \in \mathbb{R} \rbrace
    I' = \lbrace l_c' | c \in \mathbb{R} \rbrace
  \end{split}
\end{equation}

Rewrite the equations of the two collections of straight generatrix in the
point-direction form.

$l_c$ passes through the point $M_c(ac, 0, \frac{c^2}{2})$, and is parallel to 
the vector $\vec{u_c}(-a, b, -c)$.

$l_c'$ passes through the point $M_c'(ac, 0, \frac{c^2}{2})$, and is parallel to 
the vector $\vec{u_c}(a, b, c)$.

\subsection{Properties of Generatrix in Hypobolic Paraboloid}
\begin{itemize}
  \item For every point $P \in S$, each collection of straight generatrix has 
  exactly one generatrix passing through it.
  \item Generatrix from the same collection is parallel to a plane.
  \item Two generatrix from the same collection are skew with each other.
  \item Generatrix from different collections intersect.
  \item There is no straight generatrix belonging to both collections.
  \item All straight generatrix in $S$ belong to either collection.
\end{itemize}

\section{Hyperboloid of One Sheet单叶双曲面}

\subsection{Equation of Generatrix in Hyperboloid of One Sheet}

Given the standard equation of a hyperboloid of one sheet:
\begin{equation}
  \frac{x^2}{a^2} + \frac{y^2}{b^2} - \frac{z^2}{c^2} = 1
\end{equation}
\[
  (\frac{x}{a} + \frac{z}{c})(\frac{x}{a} - \frac{z}{c}) = (1 + \frac{y}{b})(1 - \frac{y}{b})
\]
\begin{equation}
  l_{s:t} =
  \begin{cases}
    s(\frac{x}{a} + \frac{z}{c}) = t(1 + \frac{y}{b}) \\
    t(\frac{x}{a} - \frac{z}{c}) = s(1 - \frac{y}{b})
  \end{cases}
\end{equation}
\begin{equation}
  l_{s:t}' =
  \begin{cases}
    s(\frac{x}{a} + \frac{z}{c}) = t(1 - \frac{y}{b}) \\
    t(\frac{x}{a} - \frac{z}{c}) = s(1 + \frac{y}{b})
  \end{cases}
\end{equation}

It is apparent that each straight generatrix one-to-one corresponds to a value of $s:t$ where

\begin{equation}
  l_{\theta} =
  \begin{cases}
    \cos\theta(\frac{x}{a} + \frac{z}{c}) = \sin\theta(1 + \frac{y}{b}) \\
    \sin\theta(\frac{x}{a} - \frac{z}{c}) = \cos\theta(1 - \frac{y}{b})
  \end{cases}
\end{equation}
\begin{equation}
  l_{\theta}' =
  \begin{cases}
    \cos\theta(\frac{x}{a} + \frac{z}{c}) = \sin\theta(1 - \frac{y}{b}) \\
    \sin\theta(\frac{x}{a} - \frac{z}{c}) = \cos\theta(1 + \frac{y}{b})
  \end{cases}
\end{equation}

\begin{remark}
  
\end{remark}

Rewrite the equations of the two collections of straight generatrix in the
point-direction form. Take $l_{\theta}$ as an example:
\[
  l_{\theta} =
  \begin{cases}
    \frac{\cos\theta}{a} x - \frac{\sin\theta}{b} y + \frac{\cos\theta}{c} z = \sin\theta \\
    \frac{\sin\theta}{a} x + \frac{\cos\theta}{b} y - \frac{\sin\theta}{c} z = \cos\theta
  \end{cases} 
\]
Let $z = 0$, then we solve the following linear system with two equations and two 
unknowns
\begin{equation} \label{eq:15}
  \begin{cases}
    \frac{\cos\theta}{a} x - \frac{\sin\theta}{b} y = \sin\theta \\
    \frac{\sin\theta}{a} x + \frac{\cos\theta}{b} y = \cos\theta
  \end{cases}  
\end{equation}
and find the only solution $x = a\sin2\theta, y = b\cos2\theta$. Therefore, the 
straight generatrix $l_{\theta}$ pass through the point 
$M_{\theta}(a\sin2\theta, b\cos2\theta, 0)$.

Besides, $l_{\theta}$ is parallel to the vector $\vec{u_{\theta}}(
\begin{vmatrix}
  -\frac{\sin\theta}{b} & \frac{\cos\theta}{c} \\
  \frac{\cos\theta}{b} & -\frac{\sin\theta}{c}
\end{vmatrix},
\begin{vmatrix}
  \frac{\cos\theta}{c} & \frac{\cos\theta}{a} \\
  -\frac{\sin\theta}{c} & \frac{\sin\theta}{a}
\end{vmatrix},
\begin{vmatrix}
  \frac{\cos\theta}{a} & -\frac{\sin\theta}{b} \\
  \frac{\sin\theta}{a} & \frac{\cos\theta}{b}
\end{vmatrix}
)$, which is \\ $\vec{u_{\theta}}(-\frac{\cos2\theta}{bc}, 
\frac{\sin2\theta}{ac}, \frac{1}{ab})$, or in better form $\vec{u_{\theta}}
(-a\cos2\theta, b\sin2\theta, c)$.

\begin{remark}
  Previously I had this doubt that why $l_\theta$ passes through the point 
  $M_\theta$ instead of any other points or more points in the $xOy$ plane. 
  Based on the equation (\ref{eq:15}), there is only one solution, which means 
  $l_\theta$ has only one intersection point with the $xOy$ plane. Geometrically 
  it also makes sense since $l_\theta$ is not parallel to the $xOy$ plane given 
  its $z$-component is non-zero.
\end{remark}

Similarly, we can find that $l_\theta'$ passes through the point $M_\theta'
(a\sin2\theta, -b\cos2\theta, 0)$, and is parallel to the vector 
$\vec{u_\theta'}(a\cos2\theta, b\sin2\theta, -c)$.

\subsection{Properties of Generatrix in Hyperboloid of One Sheet}
\begin{itemize}
  \item For every point $P \in S$, each collection of straight generatrix has 
  exactly one generatrix passing through it.
  \item Two generatrix from the same collection are skew with each other.
  \item Three generatrix from the same collection are not parallel to a plane.
  \item Generatrix from different collections are coplanar.
  \item There is no straight generatrix belonging to both collections.
  \item All straight generatrix in $S$ belong to either collection.
\end{itemize}

\section{Identification of Ruled Quadric Surface}

There are only four kinds of ruled quadric surface except those within a plane:
\begin{itemize}
  \item Quadric Cylindrical Surface
  \item Quadric Conical Surface
  \item Hypobolic Paraboloid
  \item Hyperboloid of One Sheet
\end{itemize}

We can use the characteristics of these four kinds of ruled quadric surface to 
differentiate them and identify a random ruled quadric surface:
\begin{itemize}
  \item Parallelism of generatrix: There exist parallel generatrix in 
  hyperboloid of one sheet, but any two generatrix in hypobolic paraboloid are 
  not parallel.
  \item Parallelism between generatrix and planes: 
\end{itemize}


\end{document}