\documentclass[onecolumn]{ctexart}
\usepackage[utf8]{inputenc}
\usepackage{amsmath}
\usepackage{amssymb}
\usepackage{amsthm}
\usepackage{mathtools}
\usepackage{geometry}
\usepackage{graphicx}
\usepackage{float}
\usepackage{xcolor}
\usepackage{listings}
\usepackage{indentfirst}
\usepackage{bm}
\usepackage{tikz}
\usetikzlibrary{shapes,arrows}
\geometry{a4paper,scale=0.8}

\newtheorem{definition}{Definition}
\newtheorem{theorem}{Theorem}
\newtheorem{proposition}{Proposition}
\newtheorem{lemma}{Lemma}
\newtheorem{corollary}{Corollary}
\newtheorem{remark}{Remark}
\newtheorem{example}{Example}

\title{Notes of "Surfaces in Three Dimensional Euclidean Space"}
\author{Jinxin Wang}
\date{}

\begin{document}

\maketitle

\section{Overview}

Generally speaking, 建立一个几何图形的一般方程,要找出图形上点的几何特征,然后将它转化为坐标所要满足的条件,就得到了图形的一般方程。

\section{Surface of Revolution旋转面}

\subsection{Definition of Surface of Revolution and its Geometric Elements}

\subsection{General Method to Find the Equation of a Surface of Revolution}

Given the axis of rotation $l$ and a generatrix $\varGamma$ of a surface of revolution $S$:
\begin{equation}
  M \in S \Leftrightarrow \exists M' \in \varGamma \thickspace \textnormal{s.t.} \thickspace (\vec{MM'} \perp l) \wedge (d(M, l) = d(M', l))
\end{equation}

If the equation of the axis of rotation $l$ is given in the point-direction form with $M_0$ and $\vec{u}$, then
\begin{equation}
  M \in S \Leftrightarrow \exists M' \in \varGamma \thickspace \textnormal{s.t.} \thickspace (\vec{MM'} \perp \vec{u}) \wedge (d(M, M_0) = d(M', M_0)) 
\end{equation}

\subsection{Surface of Revolution with Straight Generatrix}
\begin{itemize}
  \item cylinder
  \begin{itemize}
    \item Generation: A straight generatrix rotating by an axis parallel to the generatrix
    \item Geometric elements: the axis of rotation and the radius.
  \end{itemize}
  \item cone
  \begin{itemize}
    \item Generation: A straight generatrix rotating by an axis intersecting the generatrix
    \item Geometric elements: the axis of rotation, the vertex, and the half-angle
  \end{itemize}
  \item Rotated hyperboloid of one sheet
  \begin{itemize}
    \item Generation: A straight generatrix rotating by an axis skew with the generatrix and not perpendicular with it.
    \item Geometric elements: the axis of rotation and the generatrix
  \end{itemize}
  \item plane with a circular hole
  \begin{itemize}
    \item Generation: A straight generatrix rotating by an axis skew and perpendicular with it.
  \end{itemize}
\end{itemize}

\section{Cylindrical Surface柱面}

\subsection{General Method to Find the Equation of a Cylindrical Surface}
Given the vector $\vec{u}(u_x, u_y, u_z)$ to which the axis of rotation is parallel and the directrix $\varGamma$ whose equation is $
\begin{cases}
  F(x, y, z) = 0 \\
  G(x, y, z) = 0
\end{cases}$ of a cylindrical surface $S$, then for any point $M(x, y, z)$
\begin{equation}
  \begin{split}
    M \in S &\Leftrightarrow \exists M' \in \varGamma \thickspace \textnormal{s.t.} \thickspace \vec{MM'} \parallel \vec{u} \\
            &\Leftrightarrow \exists M' \in \varGamma \thickspace \textnormal{s.t.} \thickspace \vec{MM'} = t\vec{u}, t \in \mathbb{R} \\
            &\Leftrightarrow \exists t \in \mathbb{R} \thickspace \textnormal{s.t.} \thickspace 
            \begin{cases}
              F(x + t u_x, y + t u_y, z + t u_z) = 0 \\
              G(x + t u_x, y + t u_y, z + t u_z) = 0
            \end{cases}
  \end{split}
\end{equation}

\section{Conical Surface锥面}

\begin{definition}[Geometric Definition of Conical Surfaces]
  A conical surface consists of a series of lines passing through a same point. 
  Each line is called a generatrix. The point is called the apex of the conical 
  surface. A line intersecting with all generatrix is called a directrix of the 
  conical surface.
\end{definition}

The elements to determine a conical surface: the apex and the directrix.

\subsection{General Method to Find the Equation of a Conical Surface}
Given the apex $M_0(x_0, y_0, z_0)$ and the directrix $\varGamma$ whose equation is $
\begin{cases}
  F(x, y, z) = 0 \\
  G(x, y, z) = 0
\end{cases}$ of a conical surface $S$, then
\begin{equation}
  \begin{split}
    M \in S &\Leftrightarrow \exists M' \in \varGamma \thickspace \textnormal{s.t.} \thickspace M, M', M_0 \thickspace \textnormal{lie in the same line} \thickspace \\
            &\Leftrightarrow \exists M' \in \varGamma \thickspace \textnormal{s.t.} \thickspace \vec{OM'} = t\vec{OM} + (1 - t)\vec{OM_0}, t \in \mathbb{R} \\
            &\Leftrightarrow \exists t \in \mathbb{R} \thickspace \textnormal{s.t.} \thickspace
            \begin{cases}
              F(tx + (1 - t)x_0, ty + (1 - t)y_0, tz + (1 - t)z_0) = 0 \\
              G(tx + (1 - t)x_0, ty + (1 - t)y_0, tz + (1 - t)z_0) = 0
            \end{cases}
  \end{split}
\end{equation}

\begin{remark}
  The equation derived by the above process doesn't include the apex in its zero 
  set.
\end{remark}
\begin{remark}
  The equation derived by the above process usually has some fractions (分式) with 
  variables in their denominators, e.g. $\frac{4x^2}{z^2} - \frac{y^2}{z^2} = 1$, 
  due to the elimination of parameters. It is tempting to transform it into an 
  equation without fractions, with the benefit of including the apex in its zero 
  set. However, the transformation also has possible risk of including more 
  points not belonging to the conical surface in the zero set.

  Ex1: The equation of a conical surface $\frac{4x^2}{z^2} - \frac{y^2}{z^2} = 1$ 
  with the origin as its apex, can be transformed to $4x^2 - y^2 - z^2 = 0$. 
  Compared with the previous equation, the transformation adds $(0, 0, 0)$ (the 
  apex), and the points in the lines
  $\begin{cases}
    2x - y = 0 \\
    z = 0
  \end{cases}$, and
  $\begin{cases}
    2x + y = 0 \\
    z = 0
  \end{cases}$ to the point set of the conical surface, which is wrong.

  Ex2: The equation of a conical surface $\frac{x^2}{z^2} + \frac{y^2}{z^2} = 3$ 
  with the origin as its apex, can be transformed to $x^2 + y^2 - 3z^2 = 0$. 
  Compared with the previous equation, the transformation adds $(0, 0, 0)$ (the 
  apex) to the point set of the conical surface, which is fine.

  To determine whether such transformation is valid, wen need to work out the 
  points added to the zero set by the elimination of denominators by solving the 
  transformed equation with the denominators set to 0.
\end{remark}

\begin{proposition}
  The graph of a homogeneous equation of degree $n$ is a conical surface with 
  the origin as its apex.
\end{proposition}
\begin{proof}
  Hint:
  \begin{itemize}
    \item Since the equation is a homogeneous equation of degree $n$ (it is 
    clear that $n \geq 1$ otherwise we will have a trivial equation of $0 = 0$), 
    the origin is in its point set.
    \item For any point $M(x, y, z) \in S$, it is true that $M'(\lambda x, 
    \lambda y, \lambda z) \in S, \lambda \in \mathbb{R}$. All these points form 
    a line through the origin.
  \end{itemize}
\end{proof}

\end{document}