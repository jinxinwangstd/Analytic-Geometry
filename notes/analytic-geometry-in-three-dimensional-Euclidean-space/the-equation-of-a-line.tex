\documentclass[onecolumn]{ctexart}
\usepackage[utf8]{inputenc}
\usepackage{amsmath}
\usepackage{amssymb}
\usepackage{amsthm}
\usepackage{mathtools}
\usepackage{geometry}
\usepackage{graphicx}
\usepackage{float}
\usepackage{xcolor}
\usepackage{listings}
\usepackage{indentfirst}
\usepackage{bm}
\usepackage{tikz}
\usetikzlibrary{shapes,arrows}
\geometry{a4paper,scale=0.8}

\newtheorem{definition}{Definition}
\newtheorem{theorem}{Theorem}
\newtheorem{proposition}{Proposition}
\newtheorem{lemma}{Lemma}
\newtheorem{corollary}{Corollary}
\newtheorem{remark}{Remark}
\newtheorem{example}{Example}

\title{Notes of "The Equation of a Line"}
\author{Jinxin Wang}
\date{}

\begin{document}

\maketitle

\section{Overview}
\begin{itemize}
  \item Two types of equations of a line
  \begin{itemize}
    \item The point-direction equation of a line
    \item The general equation of a line
    \item Transformation between the two types of equations of a line
    \begin{itemize}
      \item Rmk: Transform a point-direction equation to a general equation
      \item Rmk: Transform a general equation to a point-direction equation
    \end{itemize}
  \end{itemize}
  \item Positional relationships between a line and a plane
  \item Positional relationships between two lines
  \begin{itemize}
    \item Criterion for positional relationships with the point-direction form
    \item Criterion for positional relationships with the general form
  \end{itemize}
\end{itemize}

\section{Two types of equations of a line}

\subsection{The point-direction equation of a line}

\subsection{The general equation of a line}

\subsection{Transformation between the two types of equations of a line}

\begin{remark}[Transform a point-direction equation to a general equation]
  Given $\frac{x - x_0}{u_x} = \frac{y - y_0}{u_y} = \frac{z - z_0}{u_z}$, 
  the equivalent general form is
  \[
  \begin{cases}
    \frac{x - x_0}{u_x} = \frac{y - y_0}{u_y} \\
    \frac{y - y_0}{u_y} = \frac{z - z_0}{u_z}
  \end{cases}
  \]
  \begin{equation}
    \begin{cases}
      u_y x - u_x y + u_x y_0 - u_y x_0 = 0 \\
      u_z y - u_y z + u_y z_0 - u_z y_0 = 0
    \end{cases} 
  \end{equation}
\end{remark}

\begin{remark}[Transform a general equation to a point-direction equation]
  Given the general equation of a line $l$: 
  \[
  \begin{cases}
    \pi_1: A_1 x + B_1 y + C_1 z + D_1 = 0 \\
    \pi_2: A_2 x + B_2 y + C_2 z + D_2 = 0 \\
  \end{cases}
  \]
  we can find a solution $(x_0, y_0, z_0)$ to the linear system, which is the 
  coordinate of a point on the line. Since a vector $u(u_x, u_y, u_z) \parallel l$ 
  is equivalent to $(u \parallel \pi_1) \wedge (u \parallel \pi_2)$. According to 
  the theorem of parallelism between a vector and a plane, the previous condition 
  is equivalent to 
  \[
  \begin{cases}
    A_1 u_x + B_1 u_y + C_1 u_z = 0 \\
    A_2 u_x + B_2 u_y + C_2 u_z = 0 \\
  \end{cases}
  \]
  then one solution of $u$ is
  \[
  (u_x, u_y, u_z) = 
  \left(
  \begin{vmatrix}
    B_1 & C_1 \\
    B_2 & C_2
  \end{vmatrix},
  \begin{vmatrix}
    C_1 & A_1 \\
    C_2 & A_2
  \end{vmatrix},
  \begin{vmatrix}
    A_1 & B_1 \\
    A_2 & B_2
  \end{vmatrix}
  \right)
  \]
\end{remark}

\section{Positional relationships between a line and a plane}
There are three kinds of positional relationships between a line and a plane:
\begin{itemize}
  \item 平行不重合
  \item 重合
  \item 相交
\end{itemize}

\subsection{The Point-Direction Form}

\subsection{The General Form}

Suppose the equation in the general form of a line $l$ is $
\begin{cases}
  A_1 + B_1 + C_1 + D_1 = 0 \\
  A_2 + B_2 + C_2 + D_2 = 0
\end{cases}$, and the equation of a plane $\pi$ is $A_3 + B_3 + C_3 + D_3 = 0$, 
then to determine the positional relationship between them, our method is to 
transform the problem into studying the relationship between the three planes:
\begin{proposition}
  \begin{itemize}
    \item $l \parallel \pi \wedge l \nsubseteq \pi $ $\Leftrightarrow$ $
    \begin{cases}
      A_1 + B_1 + C_1 + D_1 = 0 \\
      A_2 + B_2 + C_2 + D_2 = 0 \\
      A_3 + B_3 + C_3 + D_3 = 0
    \end{cases}$ has no solution.
    \item $l$ intersected with $\pi$ $\Leftrightarrow$ $
    \begin{cases}
      A_1 + B_1 + C_1 + D_1 = 0 \\
      A_2 + B_2 + C_2 + D_2 = 0 \\
      A_3 + B_3 + C_3 + D_3 = 0
    \end{cases}$ has a unique solution $\Leftrightarrow$ $
    \begin{vmatrix}
      A_1 & B_1 & C_1 \\
      A_2 & B_2 & C_2 \\
      A_3 & B_3 & C_3 
    \end{vmatrix} \neq 0$
    \item $l \subset \pi$ $\Leftrightarrow$ $
    \begin{cases}
      A_1 + B_1 + C_1 + D_1 = 0 \\
      A_2 + B_2 + C_2 + D_2 = 0 \\
      A_3 + B_3 + C_3 + D_3 = 0
    \end{cases}$ has infinitely many solutions.
  \end{itemize}
\end{proposition}

\section{共轴平面系}
We can approach the relationship that a line belongs to a plane in another 
direction:
\begin{proposition}
  Suppose the equation in the general form of a line $l$ is $
  \begin{cases}
    A_1 + B_1 + C_1 + D_1 = 0 \\
    A_2 + B_2 + C_2 + D_2 = 0
  \end{cases}$, and the equation of a plane $\pi$ is $A_3 + B_3 + C_3 + D_3 = 0$, 
  then $l \subset \pi$ $\Leftrightarrow$ $\exists \lambda, \mu$ which at least 
  one of them are non-zero such that
  \begin{equation}
    \lambda (A_1 + B_1 + C_1 + D_1) + \mu (A_2 + B_2 + C_2 + D_2 = 0) = A_3 + B_3 + C_3 + D_3
  \end{equation}
\end{proposition}

\begin{definition}[共轴平面系]
  
\end{definition}

\section{Positional relationships between two lines}
两直线间的位置关系总览,同时也是一种判断两条直线的位置关系的流程:
\begin{itemize}
  \item 方向向量共线
  \begin{itemize}
    \item 平行
    \item 重合
  \end{itemize}
  \item 方向向量不共线
  \begin{itemize}
    \item 相交
    \item 异面
  \end{itemize}
\end{itemize}

\subsection{Criterion for positional relationships with the point-direction form}
\begin{proposition}
  Let $l_1$ be a line passing through $M_1$ and parallel to $u_1$, and $l_2$ be a line passing through $M_2$ and parallel to $u_2$. 
  \begin{itemize}
    \item $l_1$与$l_2$重合 $\Leftrightarrow$ $(u_1 \parallel u_2) \wedge \neg(\vec{M_1M_2} \parallel u_1)$
    \item $l_1$与$l_2$平行不重合 $\Leftrightarrow$ $(u_1 \parallel u_2) \wedge (\vec{M_1M_2} \parallel u_1)$
    \item $l_1$与$l_2$相交 $\Leftrightarrow$ $\neg(u_1 \parallel u_2) \wedge ((\vec{M_1M_2}, u_1, u_2) = 0)$
    \item $l_1$与$l_2$异面 $\Leftrightarrow$ $\neg(u_1 \parallel u_2) \wedge ((\vec{M_1M_2}, u_1, u_2) \neq 0)$
  \end{itemize}
\end{proposition}
\begin{proof}
  
\end{proof}

\subsection{Criterion for positional relationships with the general form}
\begin{proposition}
  Let $l_1$ be a line with the general equation as $
  \begin{cases}
    A_1 x + B_1 y + C_1 z + D_1 = 0 \\
    A_2 x + B_2 y + C_2 z + D_2 = 0 \\
  \end{cases}$, and $l_2$ be a line with the general equation as $
  \begin{cases}
    A_3 x + B_3 y + C_3 z + D_3 = 0 \\
    A_4 x + B_4 y + C_4 z + D_4 = 0 \\   
  \end{cases}$
  \begin{itemize}
    \item $l_1 \parallel l_2$ $\Leftrightarrow$ $
    \begin{vmatrix}
      A_1 & B_1 & C_1 \\
      A_2 & B_2 & C_2 \\
      A_3 & B_3 & C_3 
    \end{vmatrix} = 
    \begin{vmatrix}
      A_1 & B_1 & C_1 \\
      A_2 & B_2 & C_2 \\
      A_4 & B_4 & C_4  
    \end{vmatrix} = 0$
    \item $l_1$与$l_2$重合 $\Leftrightarrow$ 
    \item $l_1$与$l_2$平行不重合 $\Leftrightarrow$ 
    \item $l_1$与$l_2$相交 $\Leftrightarrow$
    \item $l_1$与$l_2$异面 $\Leftrightarrow$ $
    \begin{vmatrix}
      A_1 & B_1 & C_1 & D_1 \\
      A_2 & B_2 & C_2 & D_2 \\
      A_3 & B_3 & C_3 & D_3 \\
      A_4 & B_4 & C_4 & D_4
    \end{vmatrix} \neq 0$
  \end{itemize}
\end{proposition}
\begin{proof}
  
\end{proof}

\section{Application: Find the Equation of Lines or Planes with Certain Positional Relationships}

\begin{example}
  在一个仿射坐标系中,直线$l_1$有一般方程$
  \begin{cases}
    x - y + z - 1 = 0 \\
    y + z = 0
  \end{cases}$,直线$l_2$过点$M(0, 0, -1)$,平行于向量$\vec{u}(2, 1, -2)$。平面$\pi$的方程为$x + 
  y + z = 0$。求由全体与$l_1, l_2$都相交,并且平行于$\pi$的直线所构成的曲面$S$的方程。

  Solution:
  \begin{itemize}
    \item 平面参数法
    \item 直线参数法
    \item 双直线参数法
    \item 轨迹法
  \end{itemize}
\end{example}


\end{document}