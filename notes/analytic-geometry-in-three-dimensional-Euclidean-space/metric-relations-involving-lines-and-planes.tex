\documentclass[onecolumn]{ctexart}
\usepackage[utf8]{inputenc}
\usepackage{amsmath}
\usepackage{amssymb}
\usepackage{amsthm}
\usepackage{mathtools}
\usepackage{geometry}
\usepackage{graphicx}
\usepackage{float}
\usepackage{xcolor}
\usepackage{listings}
\usepackage{indentfirst}
\usepackage{bm}
\usepackage{tikz}
\usetikzlibrary{shapes,arrows}
\geometry{a4paper,scale=0.8}

\newtheorem{definition}{Definition}
\newtheorem{theorem}{Theorem}
\newtheorem{proposition}{Proposition}
\newtheorem{lemma}{Lemma}
\newtheorem{corollary}{Corollary}
\newtheorem{remark}{Remark}
\newtheorem{example}{Example}

\title{Notes of "Metric Relations Involving Lines and Planes"}
\author{Jinxin Wang}
\date{}

\begin{document}

\maketitle

\section{Overview}

\section{度量的基础}
\begin{itemize}
  \item 向量的内积运算给出空间中两点之间距离的度量。
  \item 向量的内积和外积运算给出空间中直线之间的角度的度量。
\end{itemize}

\section{直角坐标系中图形方程的几何意义}

\subsection{平面的一般方程}

\subsection{直线的一般方程}

\subsection{共轴平面系}

\section{距离的度量}
点到直线的距离和点到平面的距离是基础,其它距离都可转化为这两种距离之一。

\subsection{点到直线的距离}

\subsection{点到平面的距离}

\subsection{平行直线之间的距离}

\subsection{直线到平行平面的距离}

\subsection{平行平面之间的距离}

\subsection{异面直线的距离与公垂线的方程}

\begin{definition}
  两异面直线的公垂线定义为与这两条直线都相交且垂直的直线。

  两异面直线之间的距离定义为它们的公垂线与这两条直线的交点的距离。
\end{definition}

求两异面直线之间的距离的方法:
\begin{itemize}
  \item 转化为两点之间的距离:利用参数设出公垂线与两直线的交点,利用公垂线的几何特性(垂直)求出交点坐标,则异面直线的距离等于两交点的距离。
  \item 转化为直线到平行平面的距离:求出过其中一条直线与另一条直线平行的平面,则异面直线距离等于直线到所求平行平面的距离。
\end{itemize}

求两异面直线的公垂线方程的方法:
\begin{itemize}
  \item 利用公垂线与异面直线的交点:同上面利用两点之间的距离求异面直线的距离的方法。
  \item 利用公垂线与异面直线决定的平面:公垂线的方向向量易求,则可求出公垂线与两异面直线各自决定的平面,则这两个平面的交线即为公垂线。
\end{itemize}

\begin{remark}
  \begin{enumerate}
    \item 利用参数设公垂线与异面直线的交点需要直线的点向式方程。一般情况下计算量较小,优先使用。
    \item 利用共轴平面系求各种平面的方法虽然不需要将直线的一般方法转化为点向式方程,但一般情况下计算量仍比利用公垂线交点的方法要大,因此一般不优先使
    用。但其中体现的几何特征还是值得掌握。
  \end{enumerate}
\end{remark}

\section{夹角的度量}

\subsection{平面与平面之间的夹角}

\subsection{直线与直线之间的夹角}

\subsection{直线与平面之间的夹角}

\end{document}