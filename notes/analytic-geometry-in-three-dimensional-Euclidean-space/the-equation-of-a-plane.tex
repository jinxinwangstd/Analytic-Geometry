\documentclass[onecolumn]{ctexart}
\usepackage[utf8]{inputenc}
\usepackage{amsmath}
\usepackage{amssymb}
\usepackage{amsthm}
\usepackage{mathtools}
\usepackage{geometry}
\usepackage{graphicx}
\usepackage{float}
\usepackage{xcolor}
\usepackage{listings}
\usepackage{indentfirst}
\usepackage{bm}
\usepackage{tikz}
\usetikzlibrary{shapes,arrows}
\geometry{a4paper,scale=0.8}

\newtheorem{definition}{Definition}
\newtheorem{theorem}{Theorem}
\newtheorem{proposition}{Proposition}
\newtheorem{lemma}{Lemma}
\newtheorem{corollary}{Corollary}
\newtheorem{remark}{Remark}
\newtheorem{example}{Example}

\title{Notes of "The Equation of a Plane"}
\author{Jinxin Wang}
\date{}

\begin{document}

\maketitle

\section{Overview}
\begin{itemize}
  \item The equation of a plane
  \item A geometric interpretation of the coefficients of the general equation of a plane
  \item Positional relationships between planes
  \item A geometric interpretation of a linear inequality with three unknowns
\end{itemize}

\section{The equation of a plane}

\begin{proposition}
  In an affine coordinate system, a plane corresponds to a linear equation with 
  three unknowns in which at least one of their coefficients are nonzero, and 
  vice versa.
\end{proposition}
\begin{proof}
  Hint: (TODO)
\end{proof}
\begin{remark}[The general equation of a plane]
  
\end{remark}

\section{A geometric interpretation of the coefficients of the general equation of a plane}
\begin{theorem}
  Given the general equation of a plane: $Ax + By + Cz + D = 0$, a vector 
  $\vec{u} = (u_x, u_y, u_z)$ is parallel to the plane if and only if
  \begin{equation}
    Au_x + Bu_y + Cu_z = 0
  \end{equation}
\end{theorem}
\begin{proof}
  Hint: (TODO)
\end{proof}

\section{Positional relationships between planes}
There are only two kinds of positional relationships between planes:
\begin{itemize}
  \item Parallel
  \item Intersected
\end{itemize}

\begin{example}
  The intersection of three planes $\pi_1: $, $\pi_2: $, $\pi_3$ is unique $\Leftrightarrow$
  $\begin{vmatrix}
    A_1 & B_1 & C_1 \\
    A_2 & B_2 & C_2 \\
    A_3 & B_3 & C_3
  \end{vmatrix} \neq 0$
\end{example}

\section{A geometric interpretation of a linear inequality with three unknowns}


\end{document}