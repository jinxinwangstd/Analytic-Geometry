\documentclass[onecolumn]{ctexart}
\usepackage[utf8]{inputenc}
\usepackage{amsmath}
\usepackage{amssymb}
\usepackage{amsthm}
\usepackage{geometry}
\usepackage{graphicx}
\usepackage{float}
\usepackage{xcolor}
\usepackage{listings}
\usepackage{indentfirst}
\usepackage{bm}
\usepackage{tikz}
\usetikzlibrary{shapes,arrows}
\geometry{a4paper,scale=0.8}

\newtheorem{definition}{Definition}
\newtheorem{theorem}{Theorem}
\newtheorem{proposition}{Proposition}
\newtheorem{lemma}{Lemma}
\newtheorem{corollary}{Corollary}
\newtheorem{remark}{Remark}
\newtheorem{example}{Example}

\DeclareMathOperator{\rank}{rank}

\title{Notes of "General Theory of Affine Coordinate Transformation"}
\author{Jinxin Wang}
\date{}

\begin{document}

\maketitle

\section{Transformation Matrix and Transformation Formulas of Vectors and Points in Two Coordinate Systems}

\begin{remark}
  由于坐标变换前后的方程描述的是同一个几何对象,因此许多几何性质在坐标变换前后保持不变,比如直线在变换后还是直线。但某些性质也会改变,如由于坐标轴单位的不同,
  不同方向上的比例关系发生了改变。
\end{remark}

\subsection{Transformation Formula of Points in Two Coordinate Systems}

Given the coordinate $(x', y', z')$ in $I'$ of a point $M$, how to find its 
corresponding coordinate $(x, y, z)$ in $I$?
\begin{itemize}
  \item $M (x', y', z')$ in $I'$ $\Leftrightarrow$ $O'M (x', y', z')$ in $I'$
  \item $O'M$ in $I$ has the coordinate $C [x', y', z']$
  \item Suppose $OO'$ in $I$, in other words, the coordinate of the point $O'$ in $I$, is $d_1, d_2, d_3$
  \item $OM = OO' + O'M$ has the coordinate $[x, y, z] = C [x', y', z'] + [d_1, d_2, d_3]$
  \item $OM(x, y, z)$ in $I$ $\Leftrightarrow$ $M (x, y, z)$ in $I$
\end{itemize}

\begin{equation}
  \begin{cases}
    x = c_{11} x' + c_{12} y' + c_{13} z' + d_1 \\
    y = c_{21} x' + c_{22} y' + c_{23} z' + d_2 \\
    z = c_{31} x' + c_{32} y' + c_{33} z' + d_3 \\
  \end{cases}
\end{equation}

\section{Transformation Formula of a Graph in Two Coordinate Systems}

\subsection{Surface}

\subsection{Curves}

\begin{example}[Change of Basis of a Line]
  Suppose the transition matrix from $I$ to $I'$ is \\ $C = 
  \begin{pmatrix}
    c_{11} & c_{12} & c_{13} \\
    c_{21} & c_{22} & c_{23} \\
    c_{31} & c_{32} & c_{33} \\
  \end{pmatrix}$. The standard equation of a line in $I$ is $\frac{x - x_0}{u_x} 
  = \frac{y - y_0}{u_y} = \frac{z - z_0}{u_z}$. To find the equation of the line 
  in $I'$, we have two methods:
  \begin{itemize}
    \item To utilize the method of change of basis of a plane, we transform the 
    standard equation of the line to the general form:
    \[
      \begin{cases}
        \frac{x - x_0}{u_x} = \frac{y - y_0}{u_y} \\
        \frac{y - y_0}{u_y} = \frac{z - z_0}{u_z}
      \end{cases}
    \]
  \end{itemize}
\end{example}


\section{Properties of Transformation Matrices}

\begin{proposition}
  All transformation matrix between two affine coordinate systems are inversible 
  matrices.
\end{proposition}

\begin{proposition}
  If the transformation matrix from $I$ to $I'$ is $C_1$, and the transformation 
  matrix from $I'$ to $I''$ is $C_2$, then the transformation matrix from $I$ to 
  $I''$ is $C_1C_2$.
\end{proposition}

\begin{corollary}
  If the transformation matrix from $I$ to $I'$ is $C$, then the transformation 
  matrix from $I'$ to $I$ is $C^{-1}$.
\end{corollary}

\begin{proposition}
  Suppose $C$ is the transformation matrix from $I$ to $I'$. $I$ and $I'$ has 
  the same orientation if and only if $|C| > 0$. $I$ and $I'$ has the opposite 
  orientation if and only if $|C| < 0$
\end{proposition}
\begin{proof}
  \[
    \begin{pmatrix}
      \vec{e'}_1 \\
      \vec{e'}_2 \\
      \vec{e'}_3 \\
    \end{pmatrix} = C^T 
    \begin{pmatrix}
      \vec{e}_1 \\
      \vec{e}_2 \\
      \vec{e}_3 \\
    \end{pmatrix}
  \]
  \[
    \begin{vmatrix}
      \vec{e'}_1 \\
      \vec{e'}_2 \\
      \vec{e'}_3 \\
    \end{vmatrix} = |C^T|
    \begin{vmatrix}
      \vec{e}_1 \\
      \vec{e}_2 \\
      \vec{e}_3 \\
    \end{vmatrix}
  \]
  \[
    (\vec{e'}_1, \vec{e'}_2, \vec{e'}_3) = |C| (\vec{e}_1, \vec{e}_2, \vec{e}_3)
  \]
\end{proof}

\section{Transformation between Cartesian Coordinate Systems}

$I \lbrack O; e_1, e_2, e_3 \rbrack$ $I' \lbrack O'; e_1', e_2', e_3' \rbrack$. The transformation matrix from $I$ to $I'$ is
\[
  C = 
  \begin{pmatrix}
    c_{11} & c_{12} & c_{13} \\
    c_{21} & c_{22} & c_{23} \\
    c_{31} & c_{32} & c_{33} \\
  \end{pmatrix}
\]
which is equivalent to
\[
  \begin{cases}
    \vec{e'}_1 = c_{11} \vec{e}_1 + c_{21} \vec{e}_2 + c_{31} \vec{e}_3 \\
    \vec{e'}_2 = c_{12} e_1 + c_{22} e_2 + c_{32} e_3 \\
    \vec{e'}_3 = c_{13} e_1 + c_{23} e_2 + c_{33} e_3 \\
  \end{cases}
\]
\[
  \vec{e'}_i \cdot \vec{e'}_j = c_{1i} c_{1j} + c_{2i} c_{2j} + c_{3i} c_{3j}
\]

Since $I'$ is a Cartesian coordinate system,
\[
  \vec{e'}_i \cdot \vec{e'}_j = \delta_{ij}
\]
\[
  c_{1i} c_{1j} + c_{2i} c_{2j} + c_{3i} c_{3j} = \delta_{ij}
\]
The above equation means that the inner product of two column vectors of $C$ is 
$0$, and the inner product between a column vector of $C$ and itself is $1$. 
Using matrix multiplication to express it:
\[
  C^T C = 
  \begin{pmatrix}
    c_{11} & c_{21} & c_{31} \\
    c_{12} & c_{22} & c_{32} \\
    c_{13} & c_{23} & c_{33} \\
  \end{pmatrix}
  \begin{pmatrix}
    c_{11} & c_{12} & c_{13} \\
    c_{21} & c_{22} & c_{23} \\
    c_{31} & c_{32} & c_{33} \\
  \end{pmatrix} = 
  \begin{pmatrix}
    \vec{e'}_1 \cdot \vec{e'}_1 & \vec{e'}_2 \cdot \vec{e'}_1 & \vec{e'}_3 \cdot \vec{e'}_1 \\
    \vec{e'}_1 \cdot \vec{e'}_2 & \vec{e'}_2 \cdot \vec{e'}_2 & \vec{e'}_3 \cdot \vec{e'}_2 \\
    \vec{e'}_1 \cdot \vec{e'}_3 & \vec{e'}_2 \cdot \vec{e'}_3 & \vec{e'}_3 \cdot \vec{e'}_3 \\
  \end{pmatrix} = 
  \begin{pmatrix}
    1 & 0 & 0 \\
    0 & 1 & 0 \\
    0 & 0 & 1 \\
  \end{pmatrix} = E
\]
which shows that $C$ is an orthogonal matrix.

\begin{remark}
  It is clear that the properties of the basis of the two Cartesian coordinate 
  systems, namely the vectors in a basis are unit vectors and they are 
  orthogonal to each other, decide that the transformation matrix between them 
  is an orthogonal matrix.
\end{remark}

Therefore, we have the following proposition:
\begin{proposition}
  The transformation matrix between two Cartesian coordinate systems is an 
  orthogonal matrix.
\end{proposition}

By the property of an inverse matrix, it also holds that
\[
  C C^T = C^T C = E
\]
which means that the inner product of two row vectors of $C$ is $0$, and the 
inner product between a row vector of $C$ and itself is $1$.

\subsection{Transformation between Two Cartesian Coordinate Systems in a Plane}

Suppose $C$ is the transformation matrix between two Cartesian coordinate systems in a plane:
\[
  C = 
  \begin{pmatrix}
    c_{11} & c_{12} \\
    c_{21} & c_{22} \\
  \end{pmatrix}
\]

According to the above proposition, $C$ is an orthogonal matrix, thus it 
satisfies two conditions:
\begin{enumerate}
  \item $c_{11}^2 + c_{12}^2 = c_{11}^2 + c_{21}^2 = c_{12}^2 + c_{22}^2 = c_{21}^2 + c_{22}^2 = 1$
  \item $c_{11} c_{12} + c_{21} c_{22} = c_{11} c_{21} + c_{12} c_{22} = 0$
\end{enumerate}

\[
  C = 
  \begin{pmatrix}
    \cos\theta & -\sin\theta \\
    \sin\theta & \cos\theta \\
  \end{pmatrix}
\]
\[
  C = 
  \begin{pmatrix}
    \cos\theta & \sin\theta \\
    \sin\theta & -\cos\theta \\
  \end{pmatrix} 
\]
in which $\theta \in \lbrack 0, 2\pi)$

\end{document}