\documentclass[onecolumn]{ctexart}
\usepackage[utf8]{inputenc}
\usepackage{amsmath}
\usepackage{amssymb}
\usepackage{amsthm}
\usepackage{geometry}
\usepackage{graphicx}
\usepackage{float}
\usepackage{xcolor}
\usepackage{listings}
\usepackage{indentfirst}
\usepackage{bm}
\usepackage{tikz}
\usetikzlibrary{shapes,arrows}
\geometry{a4paper,scale=0.8}

\newtheorem{definition}{Definition}
\newtheorem{theorem}{Theorem}
\newtheorem{proposition}{Proposition}
\newtheorem{lemma}{Lemma}
\newtheorem{corollary}{Corollary}
\newtheorem{remark}{Remark}
\newtheorem{example}{Example}

\title{Notes of "Affine Characteristics of Conic Sections"}
\author{Jinxin Wang}
\date{}

\begin{document}

\maketitle

Affine characteristics of conic sections refer to their geometric 
characteristics that are not affected by metric, such as:
\begin{itemize}
  \item Intersection between a lines and a curve
  \item Center of an ellipse or hyperbola
  \item Asymptotes of a hyperbola
  \item Open direction of a parabola
\end{itemize}

Our task in this section is to study the affine characteristics of conic 
sections using equations and their coefficients, to find the correspondance 
between an affine characteristic and an algebraic characteristic. As a result, 
we'll give algebraic definitions of these affine characteristics. The common 
ground on which we try to link an affine characteristic and an algebraic 
characteristic is the intersection between a line and a conic section.

Notice that the algebraic definitions of these affine characteristics are 
independent from the types of conic sections. In other words, for any type of 
conic sections we can determine these affine characteristics based on the 
algebraic definitions. However, the geometric interpretation of the affine 
characteristics must be discussed with the type of the conic section involved.

\section{Intersection between a Line and a Quadratic Curve}

\section{Center}

\section{Asymptotic Direction}

\section{Diameter and Conjugate}

\section{Line Tangent to Conic Sections}

\section{Open Direction of a Parabola}

\end{document}