\documentclass[onecolumn]{ctexart}
\usepackage[utf8]{inputenc}
\usepackage{amsmath}
\usepackage{amssymb}
\usepackage{amsthm}
\usepackage{mathtools}
\usepackage{geometry}
\usepackage{graphicx}
\usepackage{float}
\usepackage{xcolor}
\usepackage{listings}
\usepackage{indentfirst}
\usepackage{bm}
\usepackage{tikz}
\usetikzlibrary{shapes,arrows}
\geometry{a4paper,scale=0.8}

\newtheorem{definition}{Definition}
\newtheorem{theorem}{Theorem}
\newtheorem{proposition}{Proposition}
\newtheorem{lemma}{Lemma}
\newtheorem{corollary}{Corollary}
\newtheorem{remark}{Remark}
\newtheorem{example}{Example}

\title{Notes of "Fundamental Theorem of Affine Transformations"}
\author{Jinxin Wang}
\date{}

\begin{document}

\maketitle

\section{Overview}

\section{The Vector Transformation Decided by an Affine Transformation}

\section{Fundamental Theorem of Affine Transformations}

\begin{theorem}[Fundamental Theorem of Affine Transformations]
  Let $\pi$ be a plane.
  \begin{description}
    \item[T1] Suppose $f: \pi \to \pi$ is an affine transformation, $I = \lbrack 
    O; \vec{e}_1, \vec{e}_2 \rbrack$ is an affine coordinate system on $\pi$, 
    then $I' = \lbrack f(O); f(\vec{e}_1), f(\vec{e}_2) \rbrack$ is also an 
    affine coordinate system on $\pi$, and for all $P \in \pi$, the coordinates 
    of $P$ in $I$ are the same as the one of $f(P)$ in $I'$.
    \item[T2] Let $I = \lbrack O; \vec{e}_1, \vec{e}_2 \rbrack$ and $I' = \lbrack 
    O'; \vec{e}'_1, \vec{e}'_2 \rbrack$ be two affine coordinate systems on $\pi$. 
    There exists an mapping $f: \pi \to \pi$ as following: for all $P \in \pi$ 
    with the coordinates $(x, y)$ in $I$, let $f(P)$ be the point with the same 
    coordinates $(x, y)$ in $I'$, and $f: \pi \to \pi$ is an affine 
    transformation.
  \end{description}
\end{theorem}
\begin{proof}
  (TODO)
\end{proof}

\begin{remark}
  A conclusion from the fundamental theorem of affine transformations is that 
  for any two affine coordinate systems $I$ and $I'$ on a plane $\pi$, there 
  exists a unique affine transformation $f: \pi \to \pi$ such that $f(I) = I'$.

  \textbf{T2} proves the existence of $f: \pi \to \pi$.

  \textbf{T1} proves the uniqueness of $f: \pi \to \pi$. If there are two affine 
  transformations $f_1: \pi \to \pi$ and $f_2: \pi \to \pi$ with $f_1(I) = I'$ 
  and $f_2(I) = I'$, then for all $P \in \pi$, $f_1(P)$ and $f_2(P)$ have the 
  same coordinates in $I'$, which is the coordinates of $P$ in $I$, and thus 
  $f_1(P) = f_2(P)$. Hence $f_1 = f_2$.
\end{remark}

\end{document}